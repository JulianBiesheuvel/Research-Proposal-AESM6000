\section{Introduction and Relevance of the Project}\label{introduction}
% Introduce the general area of interest of the project, describe the state-of-the-art and the background / motivation of your project. What is/are the open problem(s) that need to be solved?
Glaciers worldwide are progressively retreating as a direct consequence of anthropogenic climate change \cite{immerzeel-2019, marzeion-2014, hugonnet-2021}. Glaciers' diminishing size threatens hydrological processes, socio-economic dynamics, and the climate on a global and local scale \cite{roberts-2022}. The combination of seasonal variations, changing climate patterns, and glacier mass balance are essential drivers in the glaciers' meltwater production \cite{viviroli-2011}. Many glaciers and ice sheets worldwide have already reached or surpassed peak water, with projections indicating that more glaciers will follow before the end of this century \cite{huss-2018}. The cumulative meltwater from these sources is the primary contributor to the global mean sea level nowadays \cite{roberts-2022}. Glaciers are markers of past climatic conditions and ongoing climate trends. Therefore, it is imperative to continue monitoring glaciers.

\textit{Future iterations will include the area of interest and more background information}

\subsection{Region of Interest}

Iceland will be the region of interest for this study. Approximately 10\% of Iceland's landmass—10.400 km2—is blanketed by ice caps \cite{bjornsson-2016}. Iceland counts $\sim$350 glaciers, of which three ice caps constitute 90\% of glacial coverage. These three ice caps are Vatnajökull ($\sim$7,700 km$^2$, $\sim$2,870 km$^3$, in the year 2021), Langjökull ($\sim$835 km$^2$, $\sim$171 km$^3$, in the year 2019), Hofsjökull ($\sim$805 km$^2$, $\sim$170 km$^3$, in the year 2020), and have surge-type outlet glaciers \cite{aalgeirsdottir-2020}. Surge-type glaciers are characterised by distinctive changes in geometry and activity over timescales that range from a few years to several decades and exhibit periodically large fluctuations in velocity. Climatologically, surge-type glaciers occur within maritime climate regions characterised by annual temperatures of ca. 0 to -10 degrees Celsius, along with a mean annual precipitation of ca. 200-2000 mm \cite{ingolfsson-2016}. Stake measurements for Iceland's three largest ice caps started in early 1990 and continue today \cite{bjornsson-2016}. To date, the Glacier Web Portal is the most extensive database of stake measurements, both regular and irregular, for Icelandic glaciers \cite{glacierwebportal}. From 1890/91 to 20218/19, the three largest ice caps experienced a total mass change of -540 $\mp$ 130 Gt \cite{aalgeirsdottir-2020}. Important to note here is that, contrary to other regions, non-surface mass balance is non-negligible for glaciers in Iceland due to the geothermal activity and active volcanos \cite{jhannesson-2020}. 

Iceland and Norway share similar latitudinal coordinates, but Iceland's annual mean temperature and precipitation are lower than Norway's \cite{orsteinsson-2013}. Yet another difference is that there are no surge-type glaciers in mainland Norway, except for Svalbard \cite{moholdt-2021}. Furthermore, unlike Iceland, geothermal or volcanic activities are absent in Norway. Despite these differences, Icelandic glaciers are still considered for this research due to the specific focus on surface mass balance, as the climatological features present close similarities.

\clearpage
\subsection{Glacier (Surface) Mass Balance}\label{glacier mass balance}
This section will outline the concept of glacier (surface) mass balance and how to measure and estimate it. 

\subsubsection{Definition}
Modelling glacier surface mass balance can help to understand the glaciers' current and future health and behaviour. A collection of mass exchange processes governs glacier mass balance; these processes add or subtract mass from the glaciers' total volume. Accumulation is the process by which a glacier gains mass, whereas ablation denotes the process by which a glacier loses mass. In the winter period, the glacier experiences a positive balance; accumulation is larger than ablation. Conversely, in the summer period, the glacier has a negative balance, and accumulation is smaller than ablation. However, it is not uncommon anymore to experience negative glacier mass loss during winter periods \cite{soheb-2020}. Moreover, the period of positive balance commences earlier and generally gets warmer in the year and extends for a prolonged duration \cite{thibert-2013}. Surface mass balance refers to a measure of change in the mass, a sum of accumulation and ablation, of a glacier or part of the glacier within a defined period \cite{cogley-2011}. The mass balance for the entire glacier is expressed in gigatons per year (Gt yr$^{-1}$ or mm w.e. y$^{-1}$). \citeauthor{van-den-broeke-2020} (\citeyear{van-den-broeke-2020}) denote the mass balance as the difference between surface mass balance and ice discharge across the grounding line, neglecting basalt melt or ground ice and assuming the ground line position is stationary. Intuitively, mass balance accounts for the total sum of all exchange processes at the ice-ocean interface (ice discharge, basalt melt) and the ice-atmosphere interface (snowfall, sublimation, meltwater runoff) \cite{van-den-broeke-2020}. Besides climate conditions, glacier topography and geometry also affect the glaciers' mass balance \cite{fujita-2008, florentine-2018, huss-2012}. Together, climatological conditions, glacier topography, and geometry introduce spatio-temporal variability in glacier mass balance, resulting in different rates of mass changes across various locations of the glacier over time. Because of this intrinsic variability, interpolation and extrapolation of observational samples introduce uncertainties that affect the interpretability of the results \cite{ahlstrm-2015}. In this study, the emphasis will be exclusively on the surface mass balance as, in many cases, it is the primary driver of glacier mass changes that account for a significant portion of the total mass budget.    

\subsubsection{Data Acquiring Methodologies}
There are several methods for measuring the glacier's surface mass balance. This paper discusses only two relevant methods: in-situ measurements and geodetic measurements. For the latter, only space-borne techniques are considered.

\paragraph{In Situ Measurements}\mbox{}\vspace{2mm} \\ 
In-situ measurements are taken at a number of individual points with stakes placed uniformly on the glacier surface so that the glacier is covered equally. Ablation and accumulation are measured through changes in surface level between two dates. That is, from the start to the end of the season. The level difference between the two dates multiplied by the near-surface density yields a point estimate of the surface mass balance \cite{strem-1993}. To date, in-situ stake measurements, complemented by snow pits, are are the most detailed and accurate source for measuring the surface mass balance spatially across a glacier. The downside of this method is that it is slow as it requires repeated measurements even for a low spatial resolution, and conditions can be challenging \cite{kaser-2003}. Because of this, in-situ measurements are scarce and often have low temporal resolution. The World Glacier Monitoring Service currently contains 157 glaciers in the world with surface mass balance point information \cite{world-glacier-monitoring-service-2024}. 

\paragraph{Geodetic Measurements}\mbox{}\vspace{2mm}  \\ 
The focus of this section is the space-born geodetic measuring techniques; other geodetic techniques are out of the scope of this research. Space-born geodetic measurements are these days widely available, with worldwide spatial coverage, and are an important resource for mass balance observations going forward. Nowadays, three well-established geodetic methods are used to estimate the glacier mass balance: DEM differencing (stereo-images and SAR interferometry), altimetry, and gravimetry. Readers are referred to \cite{berthier-2023} for a detailed discussion of these measurement techniques in practice and their strengths and weaknesses. Both DEM differencing and altimetry encounter uncertainty when converting volume measurements to mass. All three methods are considered inadequate for measuring sub-aqueous ice loss. A shared downside is that these techniques require entire glacier coverage, including high crevasses and steep regions, which may not always be possible due to the flight direction of the satellite. Another problem is that the firn/ice density needs to be approximated, which is not always possible or accurate. Moreover, due to the spatial resolution of altimetry and gravimetry, a glacier in a steady state can yield zero mass change, whereas the interpolation of point measurements might suggest otherwise. Remote sensing has opened up new ways of measuring glaciers, offering extensive data coverage in the spatial and temporal domain and reducing the reliance on in-situ measurements. Geodetic methods are a valuable complementary data source to the glaciological method with a time series of 10 years or more to check the field-based methods and possible errors \cite{kaser-2003}. 

\subsubsection{Modelling Methodologies}

\paragraph{Physical Based Models}\mbox{}\vspace{2mm} \\
Temperature index models are simplified energy balance models that assume an empirical relationship between air temperatures and melt rates of the glacier surface \cite{hock-2003}. Energy balance models quantify melt by calculating the difference between the energy input and output on a glacier surface \cite{maussion2019oggm}. The effects of different climatological conditions on the glacier surface are interdependent and directly affect the melt and its sensitivity to warming \cite{cuffey-2010}. Readers are referred to \cite{cuffey-2010, van-den-broeke-2020} for a detailed review of the energy balance model and its components.  According to \citeauthor{hock-2003} (\citeyear{hock-2003}), temperature index models are favoured over the more complicated energy balance models. That is because of the widely available air temperature data in large quantities, interpolation and forecasting possibilities of air temperature data are considered relatively easy, and the model performances are good in contempt of their simplicity. Air temperature influences the energy balance components, such as long-wave atmospheric radiation and sensible heat flux, which are the dominant factors for melt. In short, the temperature index model amounts to the snow melt, the cumulative daily average temperatures above 0 degrees Celsius over one year, called the positive degree day. The degree day factor predicts how much melt occurs per degree; this is not equal to the glacier mass balance \cite{hock-2003, maussion2019oggm}. Based on \citeauthor{hock-2003}'s (\citeyear{hock-2003}) findings, the temperature index model performs adequately under average conditions, spatially on a glacier catchment scale and temporally for a few years. \citeauthor{marzeion-2012} (\citeyear{marzeion-2012}) extended the temperature index model to account for solid precipitation to obtain the glacier surface mass balance.  

\paragraph{Machine-Learning Models}\label{machine learning}\mbox{}\vspace{2mm} \\
Machine learning practices became more common in the Earth science community in the first decade of this century \cite{dramsch-2020}. Some people started using simple linear and non-linear regression models to predict the glacier mass balance \cite{fountain-1999, aalgeirsdottir-2003} but machine learning in the context of glacier mass balance estimation started to gain traction at the beginning of this decade \cite{mutz-2022, guidicelli-2023, anilkumar-2023, diaconu-2024, ren-2024}. Further research is being undertaken to model with machine learning the mass balance of glaciers in Switzerland and Norway \cite{van-der-meer-2024} (Sjursen., 2024, personal communication); more on this in section \ref{proposed project}. In the realm of Earth sciences, the effectiveness of machine learning stems from capturing complex, often unknown, non-linear relationships between data and the target variable and generating predictive models \cite{bergen-2019}. With advancements in remote sensing over the last two decades, there is a sea of data to be explored that will only grow larger \cite{chi-2016}. Model-driven solutions like machine learning, sometimes in combination with data-driven models, can assist in extracting useful information from this data.  

\textit{TODO}

\subparagraph{Transfer Learning}\label{transfer learning}\mbox{} \\

\textit{TODO}

\subparagraph{Parameterization}\mbox{} \\

\textit{TODO}

\subparagraph{Features}\mbox{} \\

\textit{TODO}

\paragraph{Deep-Learning Models}\mbox{} \\

\textit{TODO}
\clearpage
% % \subsection{Related Work}

\subsection{Proposed Project}\label{proposed project}
This research aims to develop a machine-learning model for glacier surface mass balance in a specific region of the world. This work will be part of the larger, ambitious Mass Balance Machine project, which aims to model glacier surface mass balance globally. There are two conditions to meet for the region of interest. The first condition requires data availability, meaning enough stake data is available for the glaciers in the region considered, with date records for the hydrological year. As for the second condition, extreme cases, i.e. tropical glaciers, are out of the project's scope. \citeauthor{van-der-meer-2024} (\citeyear{van-der-meer-2024}) and Sjursen (2024, personal communication) are researching and developing machine-learning techniques to estimate glacier surface mass balance in Switzerland and Norway, respectively. The two research projects employ the machine learning model XGBoost \cite{chen-2016}. While XGBoost is not considered minimal, i.e., a small and accessible model, compared to other machine learning models, it is regarded as minimal compared to the complexity of deep learning approaches. XGBoost is a gradient tree-boosting algorithm that is especially well-suited for handling medium-sized tabular data and can outperform some deep-learning models \cite{shwartz-ziv-2022}. For a more detailed explanation of how XGBoost works and its applications, please refer to section \ref{machine learning}: \nameref{machine learning}. For this research, Sjursen's (2024, personal communication) pre-trained model will function as the base model. The goal is to transfer the learnt knowledge from the base model to a new but related domain by fine-tuning the hyper-parameters to the specific dataset, leveraging the knowledge gained during prior training. For a more detailed explanation of transfer learning and its applications, please refer to section \ref{transfer learning}: \nameref{transfer learning}.

The target data points for Sjursen's (2024, personal communication) model are mass balance stake data for 32 glaciers, including annual, summer and winter mass balance observations. The glaciers are located all over Norway, from the 60th latitude to the 70th latitude. Due to Norway's sizeable latitudinal extent, temperatures and precipitation vary across the country \cite{lussana-2018, lussana-2019}. This change in climate across the country immediately affects the glacier surface mass balance \cite{nesje-2008}. The training data for the model are topographical features obtained via OGGM, like elevation, slope, and aspect for each stake and EAR-5 Land monthly meteorological variables \cite{maussion2019oggm, munoz2019era5}. Hyper-parameter tuning was achieved through cross-validation. Two kinds of cross-validation were tested: random k-fold split and spatial blocking. The random-split approach randomly splits the dataset into k equal-sized folds. However, the downside of this approach is that training and validation sets are not guaranteed to be independent. This is because the stake data is spatially dependent. Spatial blocking alleviates the independence problem by blocking the dataset based on spatial proximity \cite{roberts-2017}. However, much of this latter part is ongoing work that has yet to be evaluated. 

To summarise, this research will investigate the effectiveness of transfer learning in improving the accuracy of surface mass balance estimate models for glaciers in a particular region of the world. To accomplish this, the architecture of an existing machine-learning model will be enhanced to enable transfer learning. No specific approach has been selected to achieve this goal, but it will be decided throughout this research. The methods for transfer learning can be divided into two groups. The first group of solutions consists of custom-made solutions like \citeauthor{sun-2022}(\citeyear{sun-2022}) and more straightforward approaches such as hyper-parameter tuning and transferability \cite{pydata}. The second group of methods involves architectures combining neural networks and XGBoost models \cite{sarkar-2022, he-2022, shietal2018}. The target and features of the model will be similar to those of Sjursen's (2024, personal communication), e.g., stake data for hydrological years, monthly metrological features, and topographical features. For cross-validation strategies in this research, consistency is maintained by employing the same approaches described by Sjursen (2024, personal communication). The goodness of fit for the model is quantified by the Root Mean Squared Error (RMSE), Mean Squared Error (MSE), Mean Absolute Error (MEA), and the correlation ($R^2$). To test and evaluate the model performance, the model will be benchmarked against Sjursen's model for Norway. Parameterisation, i.e., hyper-parameter tuning and feature engineering of the pre-trained model, and regularisation techniques are essential in the transfer learning process. This will be an important part of the research, too.  With feature engineering and importance, the objective is to gain insights to understand better what kind of topographical and metrological features are essential for good model predictions and how they relate to the spatial and temporal resolution of the data. 