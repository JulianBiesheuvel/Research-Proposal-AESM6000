\section{Research Question(s)}\label{research questions}
% What is/are the main research question(s) to be solved? There can be more than one but be focused. You can split the research question into sub-questions. The lower-level questions will, together, provide the answer to the higher-level question(s). 

% The main research question(s) should logically follow from your Introduction.

% The research questions should be Relevant (to society or academy), Anchored (in the context that you just described), Researchable (through the methods you are going to explain later) and Precise (so it can be answered in unambiguous way). 
Concluding from the goals as defined in section \ref{proposed project}, the main objectives for the research are:
\begin{itemize}
    \item To explore the applicability of transfer-learning techniques in adapting a pre-trained surface mass balance estimation model to predict surface mass balance for glaciers in a different geographical region with other climate and topographical characteristics.
    \item To evaluate the performance and generalisation ability of a surface mass balance estimation model adapted through transfer-learning for a different geographical region.
\end{itemize}

To address the objectives outlined for this research, the following supportive research questions are considered and investigated throughout this study: 

\begin{enumerate}
    \item What are the similarities and differences in environmental conditions, topographical features, and stake data characteristics between glaciers in Norway and some other geographical regions?
    \item What architectural modifications and adaptations are required for the pre-trained surface mass balance estimation model to transfer knowledge effectively?
    \item What are the most effective strategies for fine-tuning the transferred model to optimise its performance for estimating surface mass balance in the target regions?
    \item How does the performance of the transferred model compare to established large-scale models? 
    \item How do the availability, selection, and quality of auxiliary data sources, such as climate reanalysis data and topographical data, impact the transfer learning process and the performance of the transferred model?
\end{enumerate}