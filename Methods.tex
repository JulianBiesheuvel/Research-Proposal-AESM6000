\section{Methods, Tools, and Expected Results}\label{methodologies}
% In this section you should detail how you are going to answer your research questions. 

% What is the theoretical basis of the work to be undertaken? Refer to literature in which theory, framework or methods are explained here. If you are going to develop your own method, also explain it here. Are a couple of theoretical approaches going to be used together in a hybrid approach? What are the steps to be undertaken in this methodology? 

% Creating or modifying a computer model is also considered an experiment, hence the set up, documentation, and limitations should be discussed here. Do not forget to include technical details such as instruments to be used, programming language or modelling software, and why you selected those. 
To reach the objectives in section \ref{research questions}, this study starts with Sjursen's (2024, personal communication) model. It is important to have the model standalone and running so that it will be easier to adjust its architecture and apply feature selection and tuning later in the development process.  When it is at the point that the pre-trained model is adapted to the new domain space, i.e., trained for a new dataset containing mass balance stake data, the model will be evaluated on its performance; how well does it predict the target mass balance? The supportive research questions will give a structured approach to achieving these objectives, starting by gaining insights into the area of interest and identifying its climatological and topographical characteristics and how they differ from those of Norway.  These feature characteristics are important in the process of feature selection and engineering. Features deemed important for the glaciers in Norway might not be so important to the model for the new area of interest. There a good understanding is needed. 

Next, to transfer the knowledge gained by the model during the first training, the pre-trained model needs to be retrained, possibly with adjusted features and parameters. To accomplish this, the existing model must be enhanced to reuse the pre-trained XGBoost model. As described in section \ref{proposed project}, two methods are suitable for achieving this: either maintaining the original model and tweaking the parameters and retraining the model or combining a neural network with the XGBoost model.  This will be an iterative process in which both methods will be explored. Research questions 2 and 3 are closely interlinked as fine-tuning the model is inherent to the type of architecture of the model.

When the adapted model runs and is tested, its performances can be benchmarked against established large-scale models, as mentioned in section \ref{proposed project}. First, the model is tested on its predictive capabilities for a test dataset, a dataset reserved as an independent sample for evaluating the model's ability to generalise on unseen data.

Finally, for research question 5 and comparing it to research question 1, XGBoost keeps each feature's importance, i.e., the measure of each feature's contributions to the model's predictive performance. The feature importance tells us what input topographical or climatological features significantly contributed to the model predictions. This is a vital step in understanding which features are necessary to build a model capable of generalising for data within the area of interest and potentially new regions. 

During the model's development and testing, documentation of how the code works will be maintained. This study's processing and analysis will be executed in a Python-based environment. What can limit this research is the lack of (quality) data and the model architecture incapable of transferring knowledge from one model to the other.